\documentclass[11pt,letter,sans]{moderncv}
\DeclareOldFontCommand{\rm}{\normalfont\rmfamily}{\mathrm}
\DeclareOldFontCommand{\sf}{\normalfont\sffamily}{\mathsf}
\DeclareOldFontCommand{\tt}{\normalfont\ttfamily}{\mathtt}
\DeclareOldFontCommand{\bf}{\normalfont\bfseries}{\mathbf}
\DeclareOldFontCommand{\it}{\normalfont\itshape}{\mathit}
\DeclareOldFontCommand{\sl}{\normalfont\slshape}{\@nomath\sl}
\DeclareOldFontCommand{\sc}{\normalfont\scshape}{\@nomath\sc}
\DeclareRobustCommand*\cal{\@fontswitch\relax\mathcal}
\DeclareRobustCommand*\mit{\@fontswitch\relax\mathnormal}

\moderncvstyle{banking}
\moderncvcolor{blue}
%\renewcommand{\familydefault}{\sfdefault}
%\newcommand*{\orcidsocialsymbol} {https://orcid.org/sites/default/files/files/ID_symbol_B-W_128x128.png}
\usepackage[margin=0.7in,scale=0.7]{geometry}
\addtolength{\topmargin}{-0.1in}
\addtolength{\textheight}{0.1in}
\setlength{\hintscolumnwidth}{3cm}
%\setlength{\makecvtitlenamewidth}{10cm}

\usepackage{enumitem}
\setlist{noitemsep}
\setlist[itemize]{topsep=0.2em}

\usepackage[american]{babel}
\usepackage[backend=biber,style=ieee,defernumbers=true,sorting=ydnt]{biblatex}

\AtDataInput{%
  \csnumgdef{entrycount:\therefsection}{%
    \csuse{entrycount:\therefsection}+1}}

% Print the labelnumber as the total number of entries in the
% current refsection, minus the actual labelnumber, plus one
\DeclareFieldFormat{labelnumber}{\mkbibdesc{#1}}    
\newrobustcmd*{\mkbibdesc}[1]{%
  \number\numexpr\csuse{entrycount:\therefsection}+1-#1\relax}

\addbibresource{../publications.bib}

% Make the author's name bold in publications.
\DeclareNameFormat{author}{%
  \ifthenelse{\equal{\namepartfamily}{Didion}}%
    {\textbf{\ifblank{\namepartsuffix}{}{\namepartsuffix\space}#1}}%
    {\ifblank{\namepartsuffix}{}{\namepartsuffix\space}\namepartfamily}%
  \ifthenelse{\value{listcount}=1 \AND \value{liststop}=2}%
    {\space and\space}
    {\ifthenelse{\value{listcount}<\value{liststop}}%
      {\addcomma\space}
      {}
    }
}

\usepackage{longtable}
\usepackage{lastpage}
\cfoot{\addressfont\itshape\textcolor{gray}{Page \thepage\ of \pageref{LastPage}}}

\newcommand\authorname{\textbf{Didion JP}}

\setlength{\tabcolsep}{6pt}
\newenvironment{entrylistDict}{%
  \begin{tabular*}{\textwidth}{@{\extracolsep{\fill}}ll}
}{%
  \end{tabular*}
}
\newcommand{\entryDict}[2]{%
  \textbf{#1} & \parbox[t]{140mm}{%
    #2%
    \hfill%
    \vspace{\parsep}%
  }\\}

\setlength{\tabcolsep}{6pt}
\newenvironment{entrylistTwo}{%
  \begin{tabular*}{\textwidth}{@{\extracolsep{\fill}}ll}
}{%
  \end{tabular*}
}
\newcommand{\entryTwo}[3]{%
  #1&\parbox[t]{140mm}{%
    \textbf{#2}%
    \hfill%
    {\footnotesize #3}%
	\vspace{\parsep}%
  }\\}

\usepackage{ifthen}

\setlength{\tabcolsep}{6pt}
\newenvironment{entrylistTwoFour}{%
  \begin{tabular*}{\textwidth}{@{\extracolsep{\fill}}ll}
}{%
  \end{tabular*}
}
\newcommand{\entryTwoFour}[4]{%
  #1&\parbox[t]{140mm}{%
    \textbf{#2}%
    \ifthenelse{\equal{#3}{}}{ }{, }%
    #3%
    \hfill%
    {\footnotesize #4}%
	\vspace{\parsep}%
  }\\}

\newenvironment{entrylistThree}{%
  \begin{tabular*}{\textwidth}{@{\extracolsep{\fill}}ll}
}{%
  \end{tabular*}
}
\newcommand{\entryThree}[3]{%
  \parbox[t]{140mm}{%
    \textbf{#2}\\%
    \textit{#3}\vspace{\parsep}%
  } & #1 \\}

\newenvironment{entrylistFour}{%
  \begin{tabular*}{\textwidth}{@{\extracolsep{\fill}}ll}
}{%
  \end{tabular*}
}
\newcommand{\entryFour}[4]{%
  #1&\parbox[t]{140mm}{%
    \textbf{#2}%
    \hfill%
    {\footnotesize #3}\\%
	\emph{#4}%
    \vspace{\parsep}%
  }\\}

\newenvironment{entrylistFive}{%
  \begin{longtable}{@{\extracolsep{\fill}}ll}
}{%
  \end{longtable}
}
\newcommand{\entryFive}[5]{%
  \parbox[t]{170mm}{%
    \textbf{#2}%
    \hfill%
    {\footnotesize #3~|~#1}\\%
	\emph{#4}\\%
    #5\vspace{\parsep}%
  }\\}

\newenvironment{entrylistSix}{%
  \begin{tabular*}{\textwidth}{@{\extracolsep{\fill}}ll}
}{%
  \end{tabular*}
}
\newcommand{\entrySix}[4]{%
  #1&\parbox[t]{140mm}{%
    \textbf{#2}%
    \hfill\\%
    \emph{#3}\\%
    #4\vspace{\parsep}%
  }\\}

\newenvironment{entrylistSeven}{%
  \begin{longtable}{@{\extracolsep{\fill}}ll}
}{%
  \end{longtable}
}
\newcommand{\entrySeven}[4]{%
  \parbox[t]{140mm}{%
    \textbf{#2}%
    \hfill\\%
    \emph{#3}\\%
    #4\vspace{\parsep}%
  } & #1\\}

\name{John}{Didion}
\title{PhD}
%\address{street and number}{postcode city}{country}
\phone{~(919)~536~9924}
\email{johndidion@gmail.com}
\homepage{john.didion.net}
\social[linkedin]{jdidion}
\social[twitter]{jdidion}
\social[github]{jdidion}
\social[ORCiD][orcid.org/0000-0002-8111-6261]{ORCiD 0000-0002-8111-6261}
%\extrainfo{additional information}
%\photo[64pt][0.4pt]{picture}
%\quote{Some quote}


\newenvironment{mypar}
    {\setlength{\parskip}{0.5em}}
    {}

\begin{document}
\makecvtitle
    
\section{Computational Genomics Innovator}

\scalebox{.8}{\textbf{Results-Oriented Research~\textbullet~Sequencing Expertise~\textbullet~Applied Machine Learning~\textbullet~Professional Software Engineering}}

\begin{mypar}
Creative and highly productive computational biologist with a passion for leveraging high-throughput computing and advanced machine and deep learning to interpret and exploit complex genomic and clinical datasets.
\end{mypar}

\begin{itemize}
\item {Next-generation sequencing expert with broad experience in assay design, data processing, bioinformatics, and computational analytics development.}
\item {Talented engineer with 15+ years experience developing production-grade software using Python, R, SQL, C++, Ruby, and Java.}
\item {Effective and award-winning communicator who has won competitive research grants and authored >20 peer-reviewed publications.}
\item {Versatile team member who excels as a leader, collaborator, or individual contributor.}
\end{itemize}

\section{Experience}
\begin{entrylistFive}
\entryFive
{2017 to Present}
{Personal Genome Diagnostics}
{Baltimore, MD}
{Principal Bioinformatics Scientist, Research and Development}
{\vspace{-5mm}
\begin{itemize}
\item {Leading effort to develop a bioinformatics pipeline to support offering the company's IVD assays on a new sequencing platform, thereby substantially expanding market opportunities.}
\item {Developed a novel software method for NGS error correction that enables accurate low-level variant detection in liquid biopsy samples.}
\end{itemize}
}

\entryFive
{2014 to 2017}
{National Human Genome Research Institute}
{Bethesda, MD}
{Postdoctoral Fellow, Laboratory of Dr. Francis S Collins}
{\vspace{-5mm}
\begin{itemize}
\item {Leveraged a large, multi-omics dataset to investigate epigenetic mechanisms underlying regulatory variants implicated in type 2 diabetes.}
\item {Developed machine-learning approaches for imputation of missing data in multi-omics experiments.}
\item {Investigated cell-to-cell variability in pancreatic islets, including responses to environmental perturbation, using single-cell gene expression and chromatin accessibility data.}
\item {Designed a novel sequencing assay for single-molecule resolution transcriptome analysis.}
\item {Developed Atropos, user-friendly software for QC and pre-processing of NGS reads. Created reproducible benchmark pipeline using software containers for accompanying publication.}
\item {Initiated collaborative project to undestand genomic diversity of biofilm communities.}
\item {Awarded six grants, including an American Diabetes Association fellowship and an NIH K22.}
\end{itemize}
}

\entryFive
{2016 to Present}
{American Academy of Bioinformatics}
{Bethesda, MD}
{Instructor of Bioinformatics}
{\vspace{-5mm}
\begin{itemize}
\item {Developed comprehensive, open-source course materials for workshops in DNA-Seq and RNA-Seq data analysis.}
\item {Taught workshops and earned highly positive student reviews.}
\end{itemize}
}

\entryFive
{2009 to 2014}
{University of North Carolina at Chapel Hill}
{Chapel Hill, NC}
{Research Assistant, Laboratory of Dr. Fernando Pardo-Manuel de Villena}
{\vspace{-5mm}
\begin{itemize}
\item {Characterized a novel meiotic drive locus, \textit{R2d2}, and multiple modifier loci responsible for extreme transmission distortion in interspecific crosses.}
\item {Conducted a GWAS of wild mice to identify genes associated with the accumulation of Robertsonian translocations.}
\item {Developed CLASP, a software tool for validation of cell lines used in research.}
\end{itemize}
}
\entryFive
{2007 to 2008}
{Institute for Systems Biology}
{Seattle, WA}
{Computational Biology Software Engineer, Laboratory of Dr. Ruedi Aebersold}
{\vspace{-5mm}
\begin{itemize}
\item {Created TIQAM, a work flow management system to support Multiple Reaction Monitoring (MRM) proteomics experiments.}
\end{itemize}
}
\entryFive
{2004 to 2007}
{Muze, Inc.}
{Seattle, WA}
{Software Development Engineer}
{\vspace{-5mm}
\begin{itemize}
\item {Implemented key components of web services platform for purchase and distribution of digital media, including consumer management and security.}
\item {Created an intelligent installation system that decreased deployment time for the web services platform from days to less than an hour.}
\item {Improved team efficiency by implementing a code generation framework that produced a large percentage of the domain and persistence code for the web services platform.}
\end{itemize}
}
\entryFive
{2003 to 2004}
{Encyclopaedia Britannica, Online Services}
{Chicago, IL}
{Software Developer}
{\vspace{-5mm}
\begin{itemize}
\item {Developed and tested releases of several new web products, including Spanish- and Chinese- language editions of the company's flagship product.}
\item {Elimiated substantial software licensing costs by migrating marketing and e-commerce systems from outsourced to internal solutions.}
\end{itemize}
}
\entryFive
{2001 to 2003}
{ThoughtWorks, LLC}
{Chicago, IL}
{Software Developer/Consultant}
{\vspace{-5mm}
\begin{itemize}
\item {Designed, developed, and enhanced user interface, business logic, and persistence-layer components of large-scale financial software package for the commercial leasing industry.}
\item {Trained members of India development team and helped transition project to an international, round-the-clock effort.}
\item {Co-lead efforts to build web services interoperability lab. Developed Java tools for automated compatibility testing of web service runtime environments.}
\end{itemize}
}
\end{entrylistFive}

\section{Technical Skills}
\begin{entrylistDict}
\entryDict
{Programming Languages}
{\textit{Expert:} Python, R \\ \textit{Proficient:} C++, Java, SQL, Ruby}
\entryDict
{Machine Learning}
{\textit{Python:} scikit-learn, keras, networkX \\ \textit{R:} xgboost, e1071, caret}
\entryDict
{Data Science/Visualization}
{\textit{Python:} Numpy, Pandas, Seaborn \\ \textit{R:} ggplot2, other "tidyverse" packages}
\entryDict
{High-Performance Computing}
{\textit{Containerization:} Docker, Singularity \\
 \textit{Pipelines:} Nextflow, Snakemake, CWL \\
 \textit{Job Scheduling:} SGE, SLURM, LSF}
\entryDict
{Genomics}
{\textit{NGS Assays:} DNA-Seq, RNA-Seq, Methyl-Seq, ATAC-Seq, ChIP-Seq, \\ HiC, Single-Cell (10X, Fluidigm) \\
 \textit{Bioinformatics:} Samtools, BWA, GATK \\
 \textit{Other:} SNP and methylation array analysis}
\end{entrylistDict}


\section{Education}
\begin{entrylistThree}
\entryThree
{2008 to 2014~~~}
{Doctor of Philosophy, Bioinformatics and Computational Biology}
{University of North Carolina, Chapel Hill, NC}
\entryThree
{1996 to 2001~~~}
{Bachelor of Science, Computer Science}
{Northwestern University, Evanston, IL}
\end{entrylistThree}

\pagebreak

\section{Honors \& Awards}
\begin{entrylistSeven}
\entrySeven
{2017 to Present}
{NIH 1 K22 ES028024-01 BD2K Career Transition Award}
{A Big Data Approach to Learning the Type 2 Diabetes Regulome}
{Career transition award with 3 years tenure-track funding}
\entrySeven
{2017 to Present}
{American Diabetes Association Postdoctoral Fellowship}
{A Multi-Tissue and Multi-Omics Investigaton of Type 2 Diabetes}
{Postdoctoral fellowship with up to 3 years salary and research support}
\entrySeven
{2014 to 2016}
{NIH Intramural Sequencing Center Pilot Grants}
{Four separate project proposals funded}
{Institutional award with funding for sequencing services}
\entrySeven
{2014}
{Department of Health and Human Services Ignite}
{LabGenius: The Smart Lab Notebook for Scientists}
{3-month incubator program to fund innovative projects within HHS}
\entrySeven
{2014}
{Dean's Distinguished Dissertation Award (Department Nominee)}
{University of North Carolina at Chapel Hill}
{Nominee from the Bioinformatics and Computational Biology program}
\entrySeven
{2013}
{Verne Chapman Young Scientist Award}
{International Mammalian Genome Society}
{Best trainee talk at the International Mammalian Genome Conference}
\entrySeven
{2013}
{Chicago Prize}
{Complex Traits Consortium}
{Best graduate student talk at the Complex Traits Consortium meeting}
\entrySeven
{2010}
{Genome Research Award for Outstanding Poster}
{International Mammalian Genome Society}
{Outstanding poster at the International Mammalian Genome Conference}
\end{entrylistSeven}


\section{Leadership \& Service}
\begin{entrylistThree}
\entryThree
{2016 to Present}
{Certified Software Carpentry instructor}
{Co-taught mutliple workshops on biological data science to novices.}
\entryThree
{2016 to Present}
{Organizer, NHGRI Preprint Journal Club}
{Started journal club to review and provide feedback on scientific preprints.}
\entryThree
{2016 to 2017}
{Hackathon team leader}
{Lead teams in prototyping novel bioinformatics tools in multiple hackathons organized by NCBI.}
\entryThree
{2013 to Present}
{Graduate and undergraduate student mentor}
{Designed and oversaw student projects that lead to peer-reviewed publications.}
\entryThree
{2013 to 2015}
{Secretariat member (honorary), International Mammalian Genome Society}
{}
\end{entrylistThree}
\end{document}

