\documentclass[12pt,letter,sans]{moderncv}
\DeclareOldFontCommand{\rm}{\normalfont\rmfamily}{\mathrm}
\DeclareOldFontCommand{\sf}{\normalfont\sffamily}{\mathsf}
\DeclareOldFontCommand{\tt}{\normalfont\ttfamily}{\mathtt}
\DeclareOldFontCommand{\bf}{\normalfont\bfseries}{\mathbf}
\DeclareOldFontCommand{\it}{\normalfont\itshape}{\mathit}
\DeclareOldFontCommand{\sl}{\normalfont\slshape}{\@nomath\sl}
\DeclareOldFontCommand{\sc}{\normalfont\scshape}{\@nomath\sc}
\DeclareRobustCommand*\cal{\@fontswitch\relax\mathcal}
\DeclareRobustCommand*\mit{\@fontswitch\relax\mathnormal}

\moderncvstyle{banking}
\moderncvcolor{blue}
%\renewcommand{\familydefault}{\sfdefault}
%\newcommand*{\orcidsocialsymbol} {https://orcid.org/sites/default/files/files/ID_symbol_B-W_128x128.png}
\usepackage[margin=0.7in,scale=0.7]{geometry}
\addtolength{\topmargin}{-0.1in}
\addtolength{\textheight}{0.1in}
\setlength{\hintscolumnwidth}{3cm}
%\setlength{\makecvtitlenamewidth}{10cm}

\usepackage{enumitem}
\setlist{noitemsep}
\setlist[itemize]{topsep=0.2em}

\usepackage[american]{babel}
\usepackage[backend=biber,style=ieee,defernumbers=true,sorting=ydnt]{biblatex}

\AtDataInput{%
  \csnumgdef{entrycount:\therefsection}{%
    \csuse{entrycount:\therefsection}+1}}

% Print the labelnumber as the total number of entries in the
% current refsection, minus the actual labelnumber, plus one
\DeclareFieldFormat{labelnumber}{\mkbibdesc{#1}}    
\newrobustcmd*{\mkbibdesc}[1]{%
  \number\numexpr\csuse{entrycount:\therefsection}+1-#1\relax}

\addbibresource{../publications.bib}

% Make the author's name bold in publications.
\DeclareNameFormat{author}{%
  \ifthenelse{\equal{\namepartfamily}{Didion}}%
    {\textbf{\ifblank{\namepartsuffix}{}{\namepartsuffix\space}#1}}%
    {\ifblank{\namepartsuffix}{}{\namepartsuffix\space}\namepartfamily}%
  \ifthenelse{\value{listcount}=1 \AND \value{liststop}=2}%
    {\space and\space}
    {\ifthenelse{\value{listcount}<\value{liststop}}%
      {\addcomma\space}
      {}
    }
}

\usepackage{longtable}
\usepackage{lastpage}
\cfoot{\addressfont\itshape\textcolor{gray}{Page \thepage\ of \pageref{LastPage}}}

\newcommand\authorname{\textbf{Didion JP}}

\setlength{\tabcolsep}{6pt}
\newenvironment{entrylistDict}{%
  \begin{tabular*}{\textwidth}{@{\extracolsep{\fill}}ll}
}{%
  \end{tabular*}
}
\newcommand{\entryDict}[2]{%
  \textbf{#1} & \parbox[t]{140mm}{%
    #2%
    \hfill%
    \vspace{\parsep}%
  }\\}

\setlength{\tabcolsep}{6pt}
\newenvironment{entrylistTwo}{%
  \begin{tabular*}{\textwidth}{@{\extracolsep{\fill}}ll}
}{%
  \end{tabular*}
}
\newcommand{\entryTwo}[3]{%
  #1&\parbox[t]{140mm}{%
    \textbf{#2}%
    \hfill%
    {\footnotesize #3}%
	\vspace{\parsep}%
  }\\}

\usepackage{ifthen}

\setlength{\tabcolsep}{6pt}
\newenvironment{entrylistTwoFour}{%
  \begin{tabular*}{\textwidth}{@{\extracolsep{\fill}}ll}
}{%
  \end{tabular*}
}
\newcommand{\entryTwoFour}[4]{%
  #1&\parbox[t]{140mm}{%
    \textbf{#2}%
    \ifthenelse{\equal{#3}{}}{ }{, }%
    #3%
    \hfill%
    {\footnotesize #4}%
	\vspace{\parsep}%
  }\\}

\newenvironment{entrylistThree}{%
  \begin{tabular*}{\textwidth}{@{\extracolsep{\fill}}ll}
}{%
  \end{tabular*}
}
\newcommand{\entryThree}[3]{%
  \parbox[t]{140mm}{%
    \textbf{#2}\\%
    \textit{#3}\vspace{\parsep}%
  } & #1 \\}

\newenvironment{entrylistFour}{%
  \begin{tabular*}{\textwidth}{@{\extracolsep{\fill}}ll}
}{%
  \end{tabular*}
}
\newcommand{\entryFour}[4]{%
  #1&\parbox[t]{140mm}{%
    \textbf{#2}%
    \hfill%
    {\footnotesize #3}\\%
	\emph{#4}%
    \vspace{\parsep}%
  }\\}

\newenvironment{entrylistFive}{%
  \begin{longtable}{@{\extracolsep{\fill}}ll}
}{%
  \end{longtable}
}
\newcommand{\entryFive}[5]{%
  \parbox[t]{170mm}{%
    \textbf{#2}%
    \hfill%
    {\footnotesize #3~|~#1}\\%
	\emph{#4}\\%
    #5\vspace{\parsep}%
  }\\}

\newenvironment{entrylistSix}{%
  \begin{tabular*}{\textwidth}{@{\extracolsep{\fill}}ll}
}{%
  \end{tabular*}
}
\newcommand{\entrySix}[4]{%
  #1&\parbox[t]{140mm}{%
    \textbf{#2}%
    \hfill\\%
    \emph{#3}\\%
    #4\vspace{\parsep}%
  }\\}

\newenvironment{entrylistSeven}{%
  \begin{longtable}{@{\extracolsep{\fill}}ll}
}{%
  \end{longtable}
}
\newcommand{\entrySeven}[4]{%
  \parbox[t]{140mm}{%
    \textbf{#2}%
    \hfill\\%
    \emph{#3}\\%
    #4\vspace{\parsep}%
  } & #1\\}

\name{John}{Didion}
\title{PhD}
%\address{street and number}{postcode city}{country}
\phone{~(919)~536~9924}
\email{john.didion@nih.gov}
\homepage{john.didion.net}
\social[linkedin]{jdidion}
\social[twitter]{jdidion}
\social[github]{jdidion}
\social[ORCiD][orcid.org/0000-0002-8111-6261]{ORCiD 0000-0002-8111-6261}
%\extrainfo{additional information}
%\photo[64pt][0.4pt]{picture}
%\quote{Some quote}

\newenvironment{mypar}
    {\setlength{\parskip}{0.5em}}
    {}

\begin{document}
\makecvtitle

\textbf{Intrexon} \\
20358 Seneca Meadows Pkwy \\
Germantown, MD 20876 \\

\textbf{To the Team at Intrexon:} \\

As a computational biologist conducting research at the leading edge of genomics and machine learning, I believe I am well-positioned to succeed in the role of \textbf{Data Scientist} at Intrexon. \\
 
My current research as an American Diabetes Association Postdoctoral Fellow in the laboratory of NIH director Dr. Francis Collins involves developing advanced machine learning and deep learning methods for discovery of molecular biomarkers and causal mechanisms underlying regulatory genetic variation in complex diseases such as type 2 diabetes (T2D). This research supports the long-term goal of identifying novel therapeutic strategies to prevent and to treat T2D. This research leverages large, multi-omics data at both the bulk tissue and single-cell level in multiple T2D-relevant tissues. In addition, my PhD research in the laboratory of Dr. Fernando Pardo-Manuel de Villena at the University of North Carolina involved development of genetic analysis platforms and bioinformatics tools to map phenotypes in the mouse Collaborative Cross and other mammalian model organisms. In the course of this research, I have developed expertise in every phase of next-generation sequencing, from experimental design through bioinformatics pipeline development, data analysis, visualization, and publication. \\
 
I will bring to Intrexon a deep understanding of mammalian/human genetics and genomics (Computational Biology PhD, UNC Chapel Hill, 3 years postdoctoral research), a background in computer science (BS, Northwestern University) and professional software engineering (7 years experience), expertise in developing user-friendly, highly performant software and reproducible pipelines in Python (\textit{e.g.} \href{https://github.com/jdidion/atropos}{\underline{Atropos}}), and a history of scientific professionalism (20 publications, multiple funded grants, and multiple conference awards). I look forward to speaking with you further about how I can put my creativity, insight, and expertise to work at Intrexon helping to advance your drug development program. \\

Sincerely, \\

\includegraphics[width=75px]{signature.png}

John P Didion, PhD

\end{document}
